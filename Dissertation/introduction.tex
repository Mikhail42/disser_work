\chapter*{Введение}							% Заголовок
\addcontentsline{toc}{chapter}{Введение}	% Добавляем его в оглавление
Обзор, введение в тему, обозначение места данной работы в мировых исследованиях и т.п.

Первая глава состоит из ...

Вторая глава состоит из ...

Третья глава состоит из ...

\textbf{Целью} данной работы является получение оценок характеристик сильнонелинейного поверхностного волнения в жидкости конечной глубины с помощью экспериментальных и численного подходов, на примере волнения в Охотском море.

Для~достижения поставленной цели необходимо было решить следующие задачи:
\begin{enumerate}
  \item Оценить влияние нелинейности поверхностного волнения на характеристики давления на дне.

  \item Получить явную приближенную формулу для одноточечной связи между вариациями придонного давления и  смещения водной поверхности.

  \item Выявить наиболее чувствительные к появления аномально больших волн геометрические и энергетические характеристики индивидуальных волн.

  \item Разработать и реализовать программные комплексы, в качестве инструментов для работы с полученными данными.


  \item Экспериментально исследовать статистические и спектральные характеристики коротковолновых колебаний акваториях окраинных морей Дальнего Востока с использованием автономной сети мареографов.

\end{enumerate}

\textbf{Актуальность:}

Актуальность изучения сильнонелинейных процессов в прибрежной зоне, в том числе аномально больших волн обусловлена в первую очередь непредвиденными опасностями, которую они могут представлять для прибрежной инфраструктуры. В связи с расширением разведки и добычи нефти и газа в шельфовой зоне океанов и морей большую важность приобрела информация об аномально больших значениях волн, поскольку буровые установки и платформы должны эксплуатироваться при любых погодных условиях включая экстремальные. Занижение расчетных значений волнения уменьшает безопасность сооружений, а завышение увеличивает их стоимость. Большой проблемой в этом смысле для шельфа о.Сахалин является недостаточная достоверность расчетных высот волн, основанная на малом объеме наблюдаемых данных.

\textbf{Основные положения, выносимые на~защиту:}
\begin{enumerate}
  \item Представлены оценки влияния нелинейности волнения на характеристики давления на дне. Показано что линейная теория может недооценивать давление на дне более чем на 17\%, во время сильнонелинейных процессов.

  \item В рамках слабо-дисперсионной полно-нелинейной теории длинных волн и предположении о движении волны с постоянной скоростью получена явная приближенная формула для одноточечной связи между вариациями давления и смещения водной поверхности. Выполнены количественные расчеты восстановления параметров волны по показаниям донного датчика, характерным для Охотского моря.

  \item Показано, что начиная с высот солитона, равных примерно половине глубины бассейна, пространственное распределение давления на дне становится двугорбым, и давление в центре волны уменьшается по сравнению с гидростатическим.

  \item По результатам численного исследования полнонелинейных уравнений динамики идеальной несжимаемой жидкости выявлены наиболее чувствительные к появлению аномально-больших волн характеристики.

  \item Впервые получены на основе проведенных долговременных натурных наблюдений статистические и спектральные оценки характеристик ветрового волнения и зыби на шельфе о.Сахалин. Представлены статистики появления аномально больших волн и эффекты концентрации геометрических характеристик таких волн в регионе.

  \item Разработана и реализована информационная система для хранения и типовой обработки данных натурных наблюдений, также разработан и реализован программный комплекс спектрального и статистического анализа натурных данных синхронных измерений гидростатического давления.
\end{enumerate}


\textbf{Научная новизна:}
Проведенные натурные наблюдения и предложенная методика обработки и анализа данных наблюдений волнения, с помощью датчиков гидростатического давления, позволила впервые получить подробные оценки статистических и спектральных характеристик ветрового волнения и зыби на шельфе южной части о.Сахалин.

На основе продолжительных данных натурного наблюдения волнения впервые получены статистики появления аномально больших волн в данном регионе. С помощью проведения серии вычислительных экспериментов показана характеристика наиболее чувствительная к появлению аномально больших волн.

Подробно проанализировано влияние эффектов нелинейности при регистрации донным датчиком давления сильнонелинейных волновых процессов в бассейне конечной глубины  с помощью численного решения полнонелинейных уравнений идеальной жидкости, а также с помощью слабодисперсионной модели Железняка-Пелиновского.С помощью вычислительных экспериментов проанализирована крутизна обрушения волны Стокса для бассейнов различной глубины.

\textbf{Научная и практическая значимость:}
Полученные в работе оценки влияния нелинейных эффектов показывают возможность изучения сильнонелинейных процессов в прибрежной зоне, в том числе аномально больших волн с помощью датчиков донного давления. Важным практическим приложением данной работы является оценка статистических и спектральных характеристик ветрового волнения и зыби в прибрежной зоне южной части о.Сахалин, а также оценка статистических и геометрических характеристик аномально больших волн в этом регионе, которые могут быть использованы при разработке буровых,  нефтегазовых платформ или иных сооружений прибрежной инфраструктуры.

Разработанные методики обработки натурных данных и программный комплекс внедрен и используется в ФГБУН «Специализированное конструкторское бюро средств автоматизации морских исследований ДВО РАН» и в ФГБУН «Институт морской геологии и геофизики ДВО РАН» для хранения и типовой обработки данных натурных наблюдений волнения.



\textbf{Степень достоверности:}
Достоверность полученных результатов при моделировании сильнонелинейных волновых процессов обеспечивается применением строго обоснованных  методов обработки и численных методов для решения уравнений. Полученные в работе результаты по влиянию эффектов нелинейности  хорошо согласуются с результатами, полученными другими авторами.

\textbf{Апробация работы:}
Основные результаты диссертации представлялись на международных и всероссийских конференциях, среди которых:
\begin{enumerate}
  \item II Международная конференция «Геоинформатика: технологии, научные проекты». (Барнаул, 2010);
  \item V Сахалинская молодежная научная школе «Природные катастрофы: изучение, мониторинг, прогноз» (Южно-Сахалинск, 2010);
  \item Международные научно-практические конференции по графическим информационным технологиям и системам «КОГРАФ» (Нижний Новгород, 2009-2012);
  \item Генеральные Ассамблеи Европейского геофизического союза (Вена, Австрия, 2010 – 2012);
  \item XVI – XVIII Международные научно-технические конференции «Информационные системы и технологии» (Нижний Новгород, 2010 – 2013);
  \item IX – XI Международные молодежные научно-технические конференции «Будущее технической науки» (Нижний Новгород, 2010 – 2012);
  \item XI Международная научно-методическая конференция «Информатика: проблемы, методология, технологии» (Воронеж, 2011);
  
\end{enumerate}
	
Результаты диссертации неоднократно докладывались на семинарах Нижегородского государственного технического университета им. Р.Е. Алексеева и Института морской геологии и геофизики ДВО РАН.


\textbf{Личный вклад.}

Соискателем:
\begin{enumerate}
  \item Разработаны и апробированы методики обработки и анализа данных натурных наблюдений волнения, полученных с помощью датчиков гидростатического давления;
  \item Проведены оценки влияния эффектов нелинейности с помощью численного решения полнонелинейных уравнений идеальной жидкости, а также с помощью слабодисперсионной модели Железняка-Пелиновского;
  \item Получены подробные оценки статистических и спектральных характеристик ветрового волнения и зыби в прибрежной зоне южной части о.Сахалин. Отдельно получены статистики появления аномально больших волн в данном регионе;
  \item По результатам вычислительных экспериментов определены характеристики наиболее чувствительные к появлению аномально больших волн;
  \item Показаны эффекты концентрации геометрических характеристик при возникновении аномально больших волн на основе обширных данных натурных наблюдений;
  \item Все представленные в диссертации данные натурных наблюдений волнения на шельфе о.Сахалин получены при непосредственном участии автора.
\end{enumerate}

\textbf{Публикации}

Методика и результаты проведенных исследований полностью отражены в публикациях по теме диссертации. Всего по теме диссертации опубликовано 40 работ, из них:
\begin{itemize}
  \item статьи в ведущих рецензируемых изданиях, рекомендованных
ВАК – 8 (из них 2 находятся в печати);
  \item статьи в рецензируемых изданиях – 2;
  \item тезисы докладов и материалы конференций – 34.
\end{itemize}

\textbf{Список публикаций, в которых изложены основные результаты работы}

\textbf{В изданиях из перечня ВАК:}

\begin{enumerate}
  \item \textbf{Кузнецов К.И.}, Куркин А.А., Пелиновский Е.Н., Ковалев П.Д. Статистические характеристики ветрового волнения на юго-восточном побережье о. Сахалин по инструментальным измерениям 2006-2009 гг. // Известия РАН «Физика атмосферы и океана». 2014. №4.
      %выйдет в июле 2014
  \item Шамин Р.В., \textbf{Кузнецов К.И.} Об оценке опасности аномальных поверхностных волн // Вестник ДВО РАН. 2013. №3. С. 65-68.
  \item \textbf{Кузнецов К.И.} , Куркин А.А. Информационная система хранения и обработки океанологических данных// "Вестник МГОУ" серия "Естественные науки" 2010. № 2.  С.101-105.
  \item \textbf{Кузнецов К.И.}, Костенко И.С., Юдин А.В., Зарочинцев В.С. Восстановление профиля морской поверхности по записям датчиков придонного давления  // «Датчики и системы» 2013 №2, с.22-27
  \item Иволгин В.И., Ковалев Д.П., Ковалев П.Д., \textbf{Кузнецов К.И.}. Регистрация ветрового волнения донным датчиком гидростатического давления// Вестн. Тамб. ун-та, Сер. Естеств. и техн. науки. - 2011. -Т. 16.- Вып. 5.-С. 1272-1276.
  \item Зайцев А.И., Костенко И.С., \textbf{Кузнецов К.И.}, Леоненков Р.В., Гиниятулин А.Р., Панфилова Ю.А    Организация инструментальных наблюдений поверхностных волн в охотском море // «Датчики и системы» 2013 №6. с.38-44.
  \item \textbf{Кузнецов К.И.}, Пелиновский Е.Н., Куркин А.А., Зайцев А.И. Восстановление поверхностных волн по измерениям  вариаций давления на морском дне // "Вестник МГОУ" серия "Естественные науки". 2013. № 3.  %\emph{(принята к публикации, выйдет в октябре 2013 г.)}.
  \item \textbf{Кузнецов К.И.}, Зайцев А.И., Пелиновский Е.Н., Куркин А.А. Давление на дно, вызванное прохождением уединенной волны в прибрежной зоне // «Экологические системы и приборы». 2013 №
      %(\emph{принята к публикации, выйдет в октябре 2013 г)}

\textbf{в рецензируемых журналах:}

  \item Зайцев А.И., Малашенко А.Е., Костенко И.С., Пелиновский Е.Н., \textbf{Кузнецов К.И.} Регистрация волн-убийц в заливе Анива Охотского моря  Труды Нижегородского государственного технического университета им. Р.Е. Алексеева / НГТУ им.  Р.Е. Алексеева. – Нижний Новгород, 2012. №1 (94). – 366 с., стр.33-41
  \item Зайцев А.И., Костенко И.С., Леоненков Р.В., \textbf{Кузнецов К.И.}, Гиниятуллин А.Р., Панфилова Ю.А. Организация натурных наблюдений поверхностного волнения в прибрежной зоне о. Сахалин // Труды Нижегородского государственного технического университета им. Р.Е. Алексеева / НГТУ им.  Р.Е. Алексеева. – Нижний Новгород, 2012. №4 (97). – 366 с., стр. 59-68
      
\textbf{статьи в сборниках материалов:}

  \item \textbf{Кузнецов К.И.}, Куркин А.А., Ковалев Д.П., Шевченко Г.В. Характеристики ветрового волнения на западном побережье о.Сахалин // Сборник материалов IV Сахалинской молодежной школы <<Природные катастрофы: изучение, мониторинг, прогноз>>. - Южно-Сахалинск, 2010. стр. 249-255.
\end{enumerate}

а также в \textbf{34} тезисах докладов международных и всероссийских конференций.

\textbf{Свидетельства о регистрации ПО:}
\begin{enumerate}
  \item Свидетельство о регистрации программы для ЭВМ №2011610808 «Информационная система хранения и обработки гидрологических данных» правообладатель: ГОУ ВПО НГТУ им. Р.Е. Алексеева авторы: Кузнецов К.И., Куркин А.А., выдан 18.01.2011
  \item Свидетельство о регистрации программы для ЭВМ №2012614667 «Программный комплекс спектрального и статического анализа натурных данных синхронных измерений гидростатического давления»  правообладатель: ГОУ ВПО НГТУ им. Р.Е. Алексеева авторы: Кузнецов К.И., Куркин А.А., выдан \textcolor[rgb]{1.00,0.00,0.00}{12.06.2012}
\end{enumerate}

\textbf{Объем и структура работы.} Диссертация состоит из~введения, четырех глав, заключения и~двух приложений. Полный объем диссертации составляет ХХХ~страница с~ХХ~рисунками и~ХХ~таблицами. Список литературы содержит ХХХ~наименований.

\clearpage
