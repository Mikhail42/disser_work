\documentclass[11pt]{article}
\usepackage{multicol}

\usepackage{biolist}

\nofiles

\columnseprule=0.4pt
\raggedcolumns

\def\textfrac#1/#2{\leavevmode\kern.1em
\raise.5ex\hbox{\scriptsize #1}\kern-.1em
/\kern-.15em\lower.25ex\hbox{\scriptsize #2}}
\def\TextFrac#1#2{\textfrac{#1}/{#2}}

\author{A. Shipunov, V. Chub, S. Glagolev}

\title{The list of erly flowering plants from vicinities\\
of the villages Luzhki and Turovo\\
(Serpukhov district, Moscow region)}

\date{}

\begin{document}
\maketitle

\begin{enumerate}

\item[\TextFrac{2--3}{IV}]: Plants flowering mainly in April

\item[\TextFrac{1}{V}]: In the end of April---beginning of May

\item[\TextFrac{2--3}{V}]: In the middle---end of May

\item[\TextFrac{1}{VI}]: Normally flowering in June

\end{enumerate}

\bigskip

\begin{multicols}{2}
\raggedright

\SEM{Adoxaceae}

        \VID Adoxa moschatellina L. \VK \TextFrac{1}{V}

        \VID Sambucus racemosa L. \VK \TextFrac{2--3}{V}

\SEM{Apocynaceae}

        \VID  Vincetoxicum hirundinaria Medik. \VK \TextFrac{1}{VI}

\SEM{Aristolochiaceae}

        \VID Asarum europaeum L. \VK \TextFrac{1}{V}

        \VID Aristolochia clematitis L. \VK \TextFrac{2--3}{V}

\SEM{Asparagaceae}

        \VID Convallaria majalis L. \VK \TextFrac{2--3}{V}

        \VID Maianthemum bifolium (L.) F.W.Schmidt \VK \TextFrac{2--3}{V}

        \VID Polygonatum multiflorum (L.) All. \VK \TextFrac{2--3}{V}

        \VID Polygonatum odoratum (Mill.) Druce \VK \TextFrac{2--3}{V}

\SEM{Berberidaceae}

        \VID Berberis vulgaris L. \VK \TextFrac{1}{VI}

\SEM{Betulaceae}

        \VID Betula pendula Roth \VK \TextFrac{2--3}{IV}

        \VID Betula pubescens Ehrh. \VK \TextFrac{2--3}{IV}

\SEM{Boraginaceae}

        \VID Lithospermum officinale L. \VK \TextFrac{2--3}{V}

        \VID Myosotis micrantha Pall. ex Lehm. \VK \TextFrac{1}{V}

        \VID Myosotis sparsiflora Pohl \VK \TextFrac{2--3}{V}

        \VID Myosotis suaveolens Waldst. et Kit. \VK \TextFrac{2--3}{V}

        \VID Pulmonaria angustifolia L. \VK ?????, \TextFrac{1}{V}

        \VID Pulmonaria obscura Dum. \VK \TextFrac{1}{V}

\SEM{Campanulaceae}

        \VID Campanula glomerata L. \VK \TextFrac{1}{VI}

\SEM{Caprifoliaceae}

        \VID Lonicera xylosteum L. \VK \TextFrac{2--3}{V}

\SEM{Caryophyllaceae}

        \VID Alsine nemorum (L.) Schreb. \VK \TextFrac{1}{VI}

        \VID Alsine media L. \VK \TextFrac{1}{V}

        \VID Arenaria serpyllifolia L. \VK \TextFrac{1}{VI}

        \VID Cerastium holosteoides Fries \VK \TextFrac{1}{V}

        \VID Dianthus capitatus Balb. ex DC. \VK \TextFrac{1}{VI}

        \VID Herniaria glabra L. \VK \TextFrac{1}{V}

        \VID Moehringia trinervia (L.) Clairv. \VK \TextFrac{1}{VI}

        \VID Silene cucubalus Wib. \VK \TextFrac{1}{VI}

        \VID Silene nutans L. \VK \TextFrac{1}{VI}

        \VID Stellaria graminea L. \VK \TextFrac{1}{VI}

        \VID Stellaria holostea L. \VK \TextFrac{2--3}{V}

        \VID Steris viscaria (L.) Rafin. \VK \TextFrac{1}{VI}

\SEM{Celastraceae}

        \VID Euonymus verrucosa Scop. \VK \TextFrac{2--3}{V}

\SEM{Compositae}

        \VID Achillea millefolium L. \VK \TextFrac{1}{VI}

        \VID Antennaria dioica (L.) Gaertn. \VK \TextFrac{2--3}{V}

        \VID Matricaria recutita L. \VK \TextFrac{1}{VI}

        \VID Petasites spurius (Retz.) Reichenb. \VK ?????, \TextFrac{1}{V}

        \VID Taraxacum officinale Wigg. s.l. \VK \TextFrac{1}{V}

        \VID Tussilago farfara L. \VK \TextFrac{2--3}{IV}

\SEM{Crassulaceae}

        \VID Sedum acre L. \VK \TextFrac{2--3}{V}

\SEM{Cruciferae}

        \VID Alliaria petiolata (Bieb.) Cavara et Grande \VK \TextFrac{2--3}{V}

        \VID Alyssum gmelinii Jord. \VK \TextFrac{1}{V}

        \VID Arabidopsis thaliana (L.) Heynh. \VK \TextFrac{2--3}{V}

        \VID Barbarea vulgaris R.Br. \VK \TextFrac{1}{V}

        \VID Berteroa incana (L.) DC. \VK \TextFrac{2--3}{V}

        \VID Capsella bursa-pastoris (L.) Medik. \VK \TextFrac{1}{V}

        \VID Cardamine amara L. \VK \TextFrac{2--3}{V}

        \VID Cardamine pratensis L. \VK \TextFrac{2--3}{V}

        \VID Dentaria bulbifera L. \VK ?????, \TextFrac{2--3}{V}

        \VID Draba nemorosa L. \VK \TextFrac{1}{V}

        \VID Erophila verna (L.) Bess. \VK \TextFrac{1}{V}

        \VID Rorippa austriaca (Crantz) Bess. \VK \TextFrac{1}{VI}

        \VID Thlaspi arvense L. \VK \TextFrac{1}{V}

        \VID Turritis glabra L. \VK \TextFrac{1}{VI}

\SEM{Cyperaceae}

        \VID Carex acuta L. \VK \TextFrac{2--3}{V}

        \VID Carex cespitosa L. \VK \TextFrac{1}{VI}

        \VID Carex digitata L. \VK \TextFrac{2--3}{V}

        \VID Carex ericetorum Poll. \VK \TextFrac{2--3}{V}

        \VID Carex juncella (Fries) Th. Fries \VK \TextFrac{1}{VI}

        \VID Carex montana L. \VK \TextFrac{2--3}{V}

        \VID Carex nigra (L.) Reichard. \VK \TextFrac{2--3}{V}

        \VID Carex pediformis C.A.Mey. \VK \TextFrac{2--3}{V}

        \VID Carex praecox Schreb. \VK \TextFrac{1}{V}

        \VID Carex pilosa Scop. \VK \TextFrac{1}{V}

        \VID Carex vaginata Tausch \VK \TextFrac{1}{VI}

        \VID Carex vulpina L. \VK \TextFrac{2--3}{V}

        \VID Eleocharis palustris (L.) Roem. et Schult. \VK \TextFrac{1}{VI}

        \VID Eleocharis mamillata Lindb. fil. \VK \TextFrac{1}{VI}

\SEM{Equisetaceae}

        \VID Equisetum arvense L. \VK \TextFrac{1}{V}

        \VID Equisetum hyemale L. \VK \TextFrac{1}{VI}

        \VID Equisetum pratense Ehrh. \VK \TextFrac{1}{V}

        \VID Equisetum sylvaticum L. \VK \TextFrac{1}{VI}

\SEM{Ericaceae}

        \VID Vaccinium myrtillus L. \VK \TextFrac{2--3}{V}

        \VID Vaccinium vitis-idaea L. \VK \TextFrac{2--3}{V}

\SEM{Euphorbiaceae}

        \VID Euphorbia villosa Waldst. et Kit. \VK \TextFrac{2--3}{V}

        \VID Euphorbia virgata Waldst. et Kit. \VK \TextFrac{2--3}{V}

        \VID Mercurialis perennis L. \VK \TextFrac{1}{V}

\SEM{Fagaceae}

        \VID Corylus avellana L. \VK \TextFrac{2--3}{IV}

        \VID Quercus robur L. \VK \TextFrac{1}{V}

\SEM{Geraniaceae}

        \VID Erodium cicutarium (L.) L'H\'er. \VK \TextFrac{1}{VI}

        \VID Geranium pratense L. \VK \TextFrac{1}{VI}

        \VID Geranium sylvaticum L. \VK \TextFrac{1}{VI}

\SEM{Gramineae}

        \VID Alopecurus equalis Sobol. \VK \TextFrac{2--3}{V}

        \VID Alopecurus geniculatus L. \VK \TextFrac{2--3}{V}

        \VID Alopecurus pratensis L. \VK \TextFrac{2--3}{V}

        \VID Anthoxanthum odoratum L. \VK \TextFrac{2--3}{V}

        \VID Dactylis glomerata L. \VK \TextFrac{1}{VI}

        \VID Hierochlo\"e repens (Host) Beauv. \VK \TextFrac{2--3}{V}

        \VID Melica nutans L. \VK \TextFrac{2--3}{V}

        \VID Melica picta C.Koch \VK \TextFrac{2--3}{V}

        \VID Milium effusum L. \VK \TextFrac{1}{VI}

        \VID Poa annua L. \VK \TextFrac{2--3}{V}

        \VID Poa pratensis L. \VK \TextFrac{1}{VI}

        \VID Stipa pennata L. \VK \TextFrac{2--3}{V}

\SEM{Juncaceae}

        \VID Luzula pilosa (L.) Willd. \VK \TextFrac{2--3}{IV}

\SEM{Labiatae}

        \VID Ajuga genevensis L. \VK \TextFrac{1}{VI}

        \VID Ajuga reptans L. \VK \TextFrac{2--3}{V}

        \VID Galeobdolon luteum Huds. \VK \TextFrac{2--3}{V}

        \VID Glechoma hederacea L. \VK \TextFrac{2--3}{V}

        \VID Dracocephalum thymiflorum L. \VK \TextFrac{2--3}{V}

        \VID Lamium purpureum L. \VK \TextFrac{1}{V}

        \VID Salvia pratensis L. \VK \TextFrac{1}{VI}

\SEM{Leguminosae}

        \VID Astragalus danicus Retz. \VK \TextFrac{1}{VI}

        \VID Chamaecytisus ruthenicus (Fisch. et Woloszcz.) Kl\'askov\'a \VK \TextFrac{2--3}{V}

        \VID Lathyrus pisiformis L. \VK \TextFrac{1}{VI}

        \VID Lathyrus sylvestris L. \VK \TextFrac{2--3}{V}

        \VID Lathyrus tuberosus L. \VK \TextFrac{2--3}{V}

        \VID Lathyrus vernus (L.) Bernh. \VK \TextFrac{1}{V}

        \VID Medicago lupulina L. \VK \TextFrac{2--3}{V}

        \VID Trifolium hybridum L. \VK \TextFrac{1}{VI}

        \VID Trifolium montanum L. \VK \TextFrac{1}{VI}

        \VID Trifolium pratense L. \VK \TextFrac{1}{VI}

        \VID Trifolium repens L. \VK \TextFrac{1}{V}

        \VID Vicia sepium L. \VK \TextFrac{1}{V}

\SEM{Liliaceae}

        \VID Fritillaria ruthenica Wikstr. \VK \TextFrac{1}{V}

        \VID Gagea lutea (L.) Ker-Gawl. \VK \TextFrac{1}{V}

        \VID Gagea minima (L.) Ker.-Gawl. \VK \TextFrac{1}{V}

        \VID Tulipa biebersteiniana Schult. et Schult. fil. \VK \TextFrac{1}{V}

        \VID Tulipa $\times$hybrida hort. \VK ?????,  \TextFrac{1}{V}

\SEM{Melanthiaceae}

        \VID Paris quadrifolia L. \VK \TextFrac{2--3}{V}

\SEM{Oxalidaceae}

        \VID Oxalis acetosella L. \VK \TextFrac{2--3}{V}

\SEM{Papaveraceae}

        \VID Chelidonium majus L. \VK \TextFrac{1}{V}

        \VID Corydalis solida (L.) Clairv. \VK \TextFrac{1}{V}

\SEM{Plantaginaceae}

        \VID Plantago lanceolata L. \VK \TextFrac{1}{VI}

        \VID Plantago media L. \VK \TextFrac{1}{VI}

\SEM{Polygalaceae}

        \VID Polygala amarella Crantz \VK \TextFrac{1}{V}

        \VID Polygala comosa Schkuhr \VK \TextFrac{1}{VI}

        \VID Polygala vulgaris L. \VK \TextFrac{1}{VI}

\SEM{Polygonaceae}

        \VID Polygonum bistorta L. \VK \TextFrac{1}{VI}

\SEM{Primulaceae}

        \VID Androsace elongata L. \VK \TextFrac{1}{V}

        \VID Androsace filiformis Retz. \VK \TextFrac{1}{V}

        \VID Androsace septentrionalis L. \VK \TextFrac{1}{V}

        \VID Primula veris L. \VK \TextFrac{1}{V}

\SEM{Ranunculaceae}

        \VID Anemone ranunculoides L. \VK \TextFrac{1}{V}

        \VID Caltha palustris L. \VK \TextFrac{1}{V}

        \VID Ficaria verna Huds. \VK \TextFrac{1}{V}

        \VID Myosurus minimus L. \VK \TextFrac{1}{V}

        \VID Pulsatilla patens (L.) Mill. \VK \TextFrac{1}{V}

        \VID Ranunculus acris L. \VK \TextFrac{2--3}{V}

        \VID Ranunculus auricomus L. \VK \TextFrac{2--3}{V}

        \VID Ranunculus cassubicus L. \VK \TextFrac{1}{V}

        \VID Ranunculus polyanthemos L. \VK \TextFrac{2--3}{V}

        \VID Ranunculus sceleratus L. \VK \TextFrac{2--3}{V}

        \VID Trollius europaeus L. \VK \TextFrac{2--3}{V}

\SEM{Rhamnaceae}

        \VID Frangula alnus Mill. \VK \TextFrac{1}{VI}

        \VID Rhamnus cathartica L. \VK \TextFrac{1}{VI}

\SEM{Rosaceae}

        \VID Alchemilla vulgaris L. s.l. \VK \TextFrac{1}{VI}

        \VID Fragaria vesca L. \VK \TextFrac{1}{VI}

        \VID Fragaria viridis (Duch.) Weston \VK \TextFrac{1}{VI}

        \VID Geum rivale L. \VK \TextFrac{2--3}{V}

        \VID Geum urbanum L. \VK \TextFrac{1}{VI}

        \VID Malus praecox (Pall.) Borkh. \VK \TextFrac{1}{V}

        \VID Padus avium Mill. \VK \TextFrac{1}{V}

        \VID Potentilla argentea L. \VK \TextFrac{1}{VI}

        \VID Potentilla anserina L. \VK \TextFrac{2--3}{V}

        \VID Potentilla arenaria Borkh. \VK \TextFrac{1}{V}

        \VID Potentilla norvegica L. \VK \TextFrac{2--3}{V}

        \VID Potentilla reptans L. \VK \TextFrac{2--3}{V}

        \VID Prunus fruticosa Pall. \VK \TextFrac{1}{V}

        \VID Prunus spinosa L. \VK \TextFrac{1}{V}

\SEM{Salicaceae}

        \VID Populus tremula L. \VK \TextFrac{2--3}{IV}

        \VID Salix alba L. \VK \TextFrac{1}{V}

        \VID Salix aurita L. \VK \TextFrac{2--3}{IV}

        \VID Salix caprea L. \VK \TextFrac{2--3}{IV}

        \VID Salix cinerea L. \VK \TextFrac{2--3}{IV}

        \VID Salix fragilis L. \VK \TextFrac{1}{V}

        \VID Salix starkeana Willd. \VK \TextFrac{2--3}{IV}

        \VID Salix triandra L. \VK \TextFrac{2--3}{IV}

        \VID Salix viminalis L. \VK \TextFrac{2--3}{IV}

\SEM{Santalaceae}

        \VID Thesium ebracteatum Hayne \VK \TextFrac{1}{V}

\SEM{Saxifragaceae}

        \VID Chrysosplenium alternifolium L. \VK \TextFrac{2--3}{IV}

        \VID Grossularia reclinata (L.) Mill. \VK \TextFrac{2--3}{V}

        \VID Ribes nigrum L. \VK \TextFrac{2--3}{V}

\SEM{Scrophulariaceae}

        \VID Lathraea squamaria L. \VK \TextFrac{1}{V}

        \VID Veronica agrestis L. \VK \TextFrac{1}{V}

        \VID Veronica arvensis L. \VK \TextFrac{2--3}{V}

        \VID Veronica chamaedrys L. \VK \TextFrac{2--3}{V}

        \VID Veronica incana L. \VK \TextFrac{1}{VI}

        \VID Veronica prostrata L. \VK \TextFrac{2--3}{V}

        \VID Veronica verna L. \VK \TextFrac{1}{V}

\SEM{Thymelaeaceae}

        \VID Daphne mezereum L. \VK \TextFrac{1}{V}

\SEM{Ulmaceae}

        \VID Ulmus glabra Huds. \VK \TextFrac{2--3}{IV}

        \VID Ulmus laevis Pall. \VK \TextFrac{2--3}{IV}

\SEM{Violaceae}

        \VID Viola arvensis Murr. \VK \TextFrac{2--3}{V}

        \VID Viola canina L. \VK \TextFrac{2--3}{V}

        \VID Viola hirta L. \VK \TextFrac{1}{V}

        \VID Viola mirabilis L. \VK \TextFrac{1}{V}

        \VID Viola montana L. \VK \TextFrac{1}{VI}

        \VID Viola palustris L. \VK \TextFrac{1}{V}

        \VID Viola riviniana Reichenb. \VK \TextFrac{1}{V}

        \VID Viola rupestris F.W.Schmidt \VK \TextFrac{1}{V}

        \VID Viola tricolor L. \VK \TextFrac{2--3}{V}

\end{multicols}

\end{document}
